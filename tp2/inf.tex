
\documentclass[a4paper]{article}
\usepackage{amsmath}
\usepackage{graphicx}
\usepackage{float}
\renewcommand{\baselinestretch}{1}
\usepackage[margin=2cm]{geometry}

\setlength{\parindent}{0in}

\begin{document}


\begin{titlepage}
\centering
\begin{figure}[H]
    \begin{center}
        \includegraphics[width=2cm,height=2cm]{UNR_logo.jpg}
    \end{center}
\end{figure}
{\scshape\large Facultad de Ciencias Exactas, Ingenier\'ia y Agrimensura\\*
                 Licenciatura en Ciencias de la Computaci\'on\par}
\vspace{3cm}
{\scshape\LARGE An\'alisis de Lenguajes de Programaci\'on\par}
{\huge\bfseries Trabajo Pr\'actico 2\par}
\vspace{3cm}
{\Large Sotelo, Bruno\par}
\vfill
{\large 27 / 09 / 2017 \par}
\end{titlepage}

\section*{Ejercicio 1}
\begin{verbatim}

num :: Integer -> LamTerm
num n = Abs "s" (Abs "z" (num' n))
        where num' 0 = LVar "z"
              num' m = App (LVar "s") (num' (m-1))

\end{verbatim}


\section*{Ejercicio 2}
\begin{verbatim}

-- Busco la variable en la lista de variables ligadas,
-- y devuelvo la distancia a su ligadura si esta ligada,
-- -1 caso contrario.
searchFor :: Name -> [(Name,Int)] -> Int
searchFor _ []           = -1
searchFor var ((v,n):xs) = if var == v then n else searchFor var xs

-- Incremento la distancia de las variables a sus ligaduras
incv :: [(Name,Int)] -> [(Name,Int)]
incv []         = []
incv ((v,n):xs) = (v,n+1):(incv xs)


conversion :: LamTerm -> Term
conversion term = conv [] term
              where conv vars (LVar var)     = let n = searchFor var vars
                                               in if n >= 0
                                                  then (Bound n)
                                                  else (Free var)
                    conv vars (App lam lam') = ((conv vars lam)
                                                :@:
                                                (conv vars lam'))
                    conv vars (Abs var lam)  = let vars' = incv vars
                                               in (Lam (conv ((var,0):vars') lam))

\end{verbatim}


\section*{Ejercicio 3}
\begin{verbatim}

shift :: Term -> Int -> Term
shift term d = shift' term d 0
               where shift' (Bound n)  d' c = if n < c 
                                              then (Bound n)
                                              else (Bound (n+d'))
                     shift' (Free v)   d' c = (Free v)
                     shift' (l :@: l') d' c = ((shift' l d' c)
                                              :@:
                                              (shift' l' d' c))
                     shift' (Lam l)    d' c = Lam (shift' l d' (c+1))

\end{verbatim}


\newpage


\section*{Ejercicio 4}
\begin{verbatim}

subst :: Term -> Term -> Int -> Term
subst (Free var)  t i = (Free var)
subst (Bound n)   t i = if n == i then t else (Bound n)
subst (t1 :@: t2) t i = (subst t1 t i) :@: (subst t2 t i)
subst (Lam t)    t' i = Lam (subst t (shift t' 1) (i+1))

\end{verbatim}


\section*{Ejercicio 5}
\begin{verbatim}

eval :: NameEnv Term -> Term -> Term
eval env (Free var)  = case env of
                         ((n,v):nvs) -> if n == var then eval env v
                                        else eval nvs (Free var)
                         []          -> (Free var)
eval env (Bound n)   = (Bound n)
eval env (t1 :@: t2) = let t1' = eval env t1
                       in case t1' of
                           (Lam t) -> eval env (shift (subst t (shift t2 1) 0) (-1))
                           _       -> (t1' :@: (eval env t2))
eval env (Lam t)     = Lam (eval env t)

\end{verbatim}


\section*{Ejercicio 6}

La idea del algoritmo en Haskell:
\begin{verbatim}

log2' :: ((Int,Int),Int) -> Int
log2' ((a,b),c) = if (c*2 - a) <= 0 then log2' ((a,b+1),c*2) else b

log2 :: Int -> Int
log2 x = log2' ((x,0),1)

\end{verbatim}

Y el algoritmo en lambda c\'alculo:

\begin{verbatim}


def minus = \n m. m pred n


def log2' = Y (\f p. (is0 (minus (mult 2 (snd p)) (fst (fst p))))
                     (f (pair (pair (fst (fst p)) (suc (snd (fst p)))) (mult 2 (snd p))))
                     (snd (fst p)))


def log2 = \n. log2' (pair (pair n zero) (suc (zero)))

\end{verbatim}



\end{document}

