
\documentclass[a4paper]{article}
\usepackage{amsmath}
\usepackage{graphicx}
\usepackage{float}
\renewcommand{\baselinestretch}{1}
\usepackage[margin=2cm]{geometry}
\usepackage{proof}
\usepackage{amssymb}

\setlength{\parindent}{0in}

\begin{document}


\begin{titlepage}
\centering
\begin{figure}[H]
    \begin{center}
        \includegraphics[width=2cm,height=2cm]{UNR_logo.jpg}
    \end{center}
\end{figure}
{\scshape\large Facultad de Ciencias Exactas, Ingenier\'ia y Agrimensura\\*
                 Licenciatura en Ciencias de la Computaci\'on\par}
\vspace{3cm}
{\scshape\LARGE An\'alisis de Lenguajes de Programaci\'on \par}
{\huge\bfseries Trabajo Pr\'actico 3 \par}
\vspace{3cm}
{\Large Sotelo, Bruno\par}
\vfill
{\large 10 / 10 / 2017 \par}
\end{titlepage}


\section*{Ejercicio 1}
$$ S = \lambda x:B\rightarrow B\rightarrow B.\; \lambda y:B\rightarrow B.
\; \lambda z:B.\; (x\: z)(y\: z) $$

Sean:
\begin{itemize}
\item $\Gamma$ un contexto cualquiera,
\item $\Gamma_{1} = \Gamma \cup \{ x:B\rightarrow B\rightarrow B\}$,
\item $\Gamma_{2} = \Gamma_{1} \cup \{ y:B\rightarrow B\}$,
\item $\Gamma_{3} = \Gamma_{2} \cup \{ z:B\}$,
\end{itemize}
se puede dar el siguiente \'arbol:
$$ \infer[\textsc{T-Abs}]
        {\Gamma \vdash \lambda x:B\rightarrow B\rightarrow B.\;
         \lambda y:B\rightarrow B.\;\lambda z:B.\; (x\: z)(y\:z)\: :\:
         (B\rightarrow B\rightarrow B)\rightarrow (B\rightarrow B)
         \rightarrow B \rightarrow B}
        {\infer[\textsc{T-Abs}]
              {\Gamma_{1} \vdash \lambda y:B\rightarrow B.\; \lambda z:B
               .\; (x\: z)(y\: z)\: :\: (B\rightarrow B) \rightarrow B
               \rightarrow B}
              {\infer[\textsc{T-Abs}]
                    {\Gamma_{2} \vdash \lambda z:B.\; (x\: z)(y\: z)\:
                     :\: B\rightarrow B}
                    {\infer[\textsc{T-App}]
                          {\Gamma_{3} \vdash (x\: z)(y\: z)\: :\: B}
                          {\infer[\textsc{T-App}]
                                {\Gamma_{3} \vdash x\: z\: :\:
                                 B\rightarrow B}
                                {\infer[\textsc{T-Var}]
                                      {\Gamma_{3} \vdash x\: :\:
                                       B\rightarrow B\rightarrow B}{}
                                 & \infer[\textsc{T-Var}]
                                      {\Gamma_{3} \vdash z\: :\: B}{}}
                           & \infer[\textsc{T-App}]
                                {\Gamma_{3} \vdash y\: z\: :\: B}
                                {\infer[\textsc{T-Var}]
                                      {\Gamma_{3} \vdash y\: :\:
                                       B \rightarrow B}{}
                                 & \infer[\textsc{T-Var}]
                                      {\Gamma_{3} \vdash z\: :\: B}{}}
                           }
                    }
              }
        } $$




\section*{Ejercicio 2}
\begin{enumerate}
\item $infer$ retorna el tipo $Either\; String\; Type$ ya que de esta
        manera puede devolver tanto un mensaje de error ($Left\; String$)
        como un tipo si el t\'ermino est\'a bien formado ($Right\; Type$).
\item La funci\'on $\gg =$ realiza un llamado a la funci\'on $either$, y en
        este contexto lo que hace es: Si el argumento $v\: :\: Either\;
        String\; Type$ pasado es un mensaje de error (tiene la forma
        $Left\; msg$), entonces devuelve este mensaje. De lo contrario,
        $v$ es de la forma $Right\; type$, entonces devuelve el resultado
        de aplicar $f\: :\: Type\rightarrow Either\; String\; Type$ (el
        otro argumento que toma $\gg =$) a $v$.
\end{enumerate}




\section*{Ejercicio 3}
Derivaci\'on de tipo para $(\textbf{let}\; z = ((\lambda x:B.\: x)\;
\textbf{as}\; B\rightarrow B)\; \textbf{in}\; z)\; \textbf{as}\;
B\rightarrow B$ \\
Sean:
\begin{itemize}
    \item $\Gamma$ un contexto cualquiera,
    \item $\Gamma_{1} = \Gamma \cup \{ x:B\}$,
    \item $\Gamma_{2} = \Gamma \cup \{ z:B\rightarrow B\}$,
\end{itemize}
el \'arbol de derivaci\'on es:
$$ \infer[\textsc{T-Ascribe}]
         {\Gamma \vdash (\textbf{let}\; z = ((\lambda x:B.\: x)\;
          \textbf{as}\; B\rightarrow B)\; \textbf{in}\; z)\;
          \textbf{as}\; B\rightarrow B\: :\: B\rightarrow B}
         {\infer[\textsc{T-Let}]
                {\Gamma \vdash \textbf{let}\; z = ((\lambda x:B.\:
                 x)\; \textbf{as}\; B\rightarrow B)\; \textbf{in}\;
                 z\: :\: B\rightarrow B}
                {\infer[\textsc{T-Ascribe}]
                       {\Gamma \vdash (\lambda x:B.\: x)\;
                        \textbf{as}\; B\rightarrow B\: :\:
                        B \rightarrow B}
                       {\infer[\textsc{T-Abs}]
                              {\Gamma \vdash \lambda x:B.\: x\:
                               :\: B\rightarrow B}
                              {\infer[\textsc{T-Var}]
                                     {\Gamma_{1} \vdash x:B}{}}
                       }
                 & \infer[\textsc{T-Var}]
                         {\Gamma_{2} \vdash z:B\rightarrow B}{}}
         }
$$
       
               


\section*{Ejercicio 7}
Reglas de evaluaci\'on para pares:
\begin{align}
    \tag{\textsc{E-PairF}}
    \infer[]
         {(t_{1},\: t_{2}) \rightarrow (t'_{1},\: t_{2})}
         {t_{1} \rightarrow t'_{1}} \\
    \nonumber \\
    \tag{\textsc{E-PairS}}
    \infer[]
         {(v,\: t) \rightarrow (v,\: t')}
         {t \rightarrow t'} \\
    \nonumber \\
    \tag{\textsc{E-Fst}}
    \infer[]
         {\textbf{fst}\; t \rightarrow \textbf{fst}\; t'}
         {t \rightarrow t'} \\
    \nonumber \\
    \tag{\textsc{E-Snd}}
    \infer[]
         {\textbf{snd}\; t \rightarrow \textbf{snd}\; t'}
         {t \rightarrow t'} \\
    \nonumber \\
    \tag{\textsc{E-FstApp}}
    \textbf{fst}\; (v_{1},v_{2}) \rightarrow v_{1} \\
    \nonumber \\
    \tag{\textsc{E-SndApp}}
    \textbf{snd}\; (v_{1},v_{2}) \rightarrow v_{2}
\end{align}
                              



\section*{Ejercicio 9}
Derivaci\'on de tipo para $\text{fst}\; (\text{unit as Unit},
\lambda x:(B,B).\: \text{snd}\; x)$ \\
Sean $\Gamma$ contexto cualquiera, $\Gamma' = \Gamma \cup
\{ x:(B,B)\}$, el \'arbol de derivaci\'on es:
$$ \infer[\textsc{T-Fst}]
        {\Gamma \vdash \text{fst}\; (\text{unit as Unit},
         \lambda x:(B,B).\; \text{snd}\; x)\: :\: \text{Unit}}
        {\infer[\textsc{T-Pair}]
               {\Gamma \vdash (\text{unit as Unit},
                \lambda x:(B,B).\; \text{snd}\; x)\: :\:
                (\text{Unit}, (B,B)\rightarrow B)}
               {\infer[\textsc{T-Ascribe}]
                      {\Gamma \vdash \text{unit as Unit}\:
                       :\: \text{Unit}}
                      {\infer[\textsc{T-Unit}]
                             {\Gamma \vdash \text{unit : Unit}}
                             {}}
                & \infer[\textsc{T-Abs}]
                        {\Gamma \vdash \lambda x:(B,B).\;
                         \text{snd}\: x\: :\: (B,B) \rightarrow B}
                        {\infer[\textsc{T-Snd}]
                               {\Gamma' \vdash \text{snd}\: x\:
                                :\: B}
                               {\infer[\textsc{T-Var}]
                                      {\Gamma' \vdash x:(B,B)}{}}
                        }
                }
        }
$$

                        



\end{document}

